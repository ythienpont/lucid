\documentclass{article}
\usepackage[landscape]{geometry}
\usepackage{url}
\usepackage{multicol}
\usepackage{amsmath}
\usepackage{esint}
\usepackage{amsfonts}
\usepackage{tikz}
\usetikzlibrary{decorations.pathmorphing}
\usepackage{amsmath,amssymb}

\usepackage{colortbl}
\usepackage{xcolor}
\usepackage{mathtools}
\usepackage{amsmath,amssymb}
\usepackage{enumitem}
\makeatletter

\newcommand*\bigcdot{\mathpalette\bigcdot@{.5}}
\newcommand*\bigcdot@[2]{\mathbin{\vcenter{\hbox{\scalebox{#2}{$\m@th#1\bullet$}}}}}
\makeatother

\title{Lucid Cheat Sheet}
\usepackage[utf8]{inputenc}

\advance\topmargin-.8in
\advance\textheight3in
\advance\textwidth3in
\advance\oddsidemargin-1.5in
\advance\evensidemargin-1.5in
\parindent0pt
\parskip2pt
\newcommand{\hr}{\centerline{\rule{3.5in}{1pt}}}
%\colorbox[HTML]{e4e4e4}{\makebox[\textwidth-2\fboxsep][l]{texto}
\begin{document}

\begin{center}{\huge{\textbf{Lucid Cheat Sheet}}}\\
\end{center}
\begin{multicols*}{3}

\tikzstyle{mybox} = [draw=black, fill=white, very thick,
    rectangle, rounded corners, inner sep=10pt, inner ysep=10pt]
\tikzstyle{fancytitle} =[fill=black, text=white, font=\bfseries]

\tikzstyle{mybox} = [draw=black, fill=white, very thick,
    rectangle, rounded corners, inner sep=10pt, inner ysep=10pt]
\tikzstyle{fancytitle} =[fill=black, text=white, font=\bfseries]

%------------ Event Declarations ---------------
\begin{tikzpicture}
\node [mybox] (box){%
    \begin{minipage}{0.45\textwidth}
\texttt{event foo(type1 id1, type2 id2, ..., typeN idN);}
\vspace{5pt}

Declares an event named \texttt{foo} that takes arguments \texttt{id1, id2, ..., idN} of types \texttt{type1, type2, ..., typeN}.
    \end{minipage}
};
\node[fancytitle, right=10pt] at (box.north west) {Event Declarations};
\end{tikzpicture}

%------------ Event Values ---------------
\begin{tikzpicture}
\node [mybox] (box){%
    \begin{minipage}{0.45\textwidth}
\texttt{event x = foo(arg1, arg2, ..., argN);}
\vspace{5pt}

Creates an event value of type \texttt{foo} containing \texttt{arg1, arg2, ..., argN} and stores it in variable \texttt{x}.
    \end{minipage}
};
\node[fancytitle, right=10pt] at (box.north west) {Event Values};
\end{tikzpicture}

%------------ Event Generation ---------------
\begin{tikzpicture}
\node [mybox] (box){%
    \begin{minipage}{0.55\textwidth}
Events are generated using \textbf{generate} statements to serialize events and send them to queues. Types include:

\begin{itemize}[leftmargin=10pt]
\item \texttt{generate\_port(n, x);} \\ 
Serializes event \texttt{x} into a packet and sends it to port \texttt{n}.
\item \texttt{generate\_ports(g, x);} \\ 
Sends event \texttt{x} to all ports in group \texttt{g}.
\item \texttt{generate(x);} \\ 
Queues event \texttt{x} for recirculation on the current switch
\end{itemize}
    \end{minipage}
};
\node[fancytitle, right=10pt] at (box.north west) {Event Generation};
\end{tikzpicture}

%------------ Packet Events ---------------
\begin{tikzpicture}
\node [mybox] (box){%
    \begin{minipage}{0.55\textwidth}
\texttt{packet event foo(type1 id1, type2 id2, ..., typeN idN);}
\vspace{5pt}

Defines a packet-based event \texttt{foo} where arguments \texttt{id1, id2, ..., idN} are packed directly into the packet. Example:

\texttt{packet event eth\_ip(eth\_hdr\_t eth, ip\_hdr\_t ip, Payload.t pl);}

Enables creation of standard IP packets but requires custom parsers for deserialization.
    \end{minipage}
};
\node[fancytitle, right=10pt] at (box.north west) {Packet Events};

\end{tikzpicture}
\end{multicols*}
\end{document}
